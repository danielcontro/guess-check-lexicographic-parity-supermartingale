\documentclass[a4paper,12pt, english]{article}

\usepackage{import}
\usepackage{amsmath}
\usepackage{amssymb}
\usepackage{amsfonts}
\usepackage{stmaryrd}
\usepackage{forest}

\author{Daniel Eduardo Contro}
\date{A.Y. 2024/2025}
\title{TreeLexPSM}


\newcommand{\Implies}{\Rightarrow}
\DeclareMathOperator{\Dom}{dom}
\DeclareMathOperator{\Cod}{cod}
\DeclareMathOperator{\Even}{even}
\DeclareMathOperator{\Odd}{odd}
\DeclareMathOperator{\Post}{Post}
\DeclareMathOperator{\Prob}{Pr}


\begin{document}

\maketitle

\section{Introduction}

In order to improve the expressiveness of the original \textit{LexPSM} template, we introduce a
generalization of it called \textit{TreeLexPSM}.
Given a reactive system $System^\phi=\langle S, TI, Init^\phi, \mathcal{G}^\phi \rangle$ and a
parity automaton $\mathcal{A} = \langle Q, \Sigma, \delta, q_0, c \rangle$ with
$c : Q \rightarrow \mathbb{N}_{<d}$ the associated coloring function, we define the product
as $Product^\phi=\langle S \times Q, \Sigma, \mathcal{G}^\phi_\times, (q_0, Init^\phi), c \rangle$
where $\mathcal{G}^\phi_\times = "\delta \times \mathcal{G}^\phi"$,
A \textit{TreeLexPSM} is a family $\Pi = (\Pi_p)_{p \in \mathbb{N}_{<d}}$ of complete binary decision trees
of height $h$, where every node is associated to a linear constraint and each leaf has a unique
\textit{LexPSM} related to it.

This induces a family $\mathcal{P}=(\mathcal{P}_p)_{p \in \mathbb{N}_{<d}}$ of families
$\mathcal{P}_p=(P_{p,n})_{n \in \mathbb{N}_{<2^h}}$, where any family $P_{p,i}$ can possibly be empty,
which is a partition of the state space $S \times Q$.
Furthermore every $P_{p,i} \in P_p$ is associated with a
\textit{LexPSM} $\lambda_{p,i} : S \to \mathbb{R}^d$.

\begin{figure}[h!]
	\begin{center}
		\begin{forest}
			[$n_0$
				[$n_1$ [$n_3$ [$l_0$] [$l_1$] ] [$n_4$ [$l_2$] [$l_3$] ] ]
					[$n_2$ [$n_5$ [$l_4$] [$l_5$] ] [$n_6$ [$l_6$] [$l_7$] ] ]
			]
		\end{forest}
		\caption{Example of a \textit{TreeLexPSM} $\tau_0$ for parity level 0 of height 3.}
	\end{center}
\end{figure}

\section{Proof Rule}

The introduction of this new template requires the rewriting of the proof rule for the synthesis of
a \textit{TreeLexPSM}.
The original proof rule for \textit{LexPSM} is as follows:
\begin{center}
	\begin{figure}[h!]
		\begin{tabular}{l}
			$\forall \phi : I(q_0, Init^\phi) \ge 0$                                                                                 \\
			$\forall \phi \forall x \in S \times Q \forall (g, U) \in \mathcal{G}^\phi_\times \forall u \in U :$                     \\
			\qquad $TI(x) \land I(x) \ge 0 \land g(x) \Implies I(u(x)) \ge 0$                                                        \\
			$\forall \phi \forall \Even p \in \mathbb{N}_{<d} \forall x \in S \times Q \forall (g, U) \in \mathcal{G}^\phi_\times :$ \\
			\qquad $O_p(x) \land TI(x) \land I(x) \ge 0 \land g(x) \Implies \Post V^\phi(x) \preceq^{\varepsilon}_{p} V^\phi(x)$     \\
			$\forall \phi \forall \Odd p \in \mathbb{N}_{<d} \forall x \in S \times Q \forall (g, U) \in \mathcal{G}^\phi_\times :$  \\
			\qquad $O_p(x) \land TI(x) \land I(x) \ge 0 \land g(x) \Implies \Post V^\phi(x) \prec^{\varepsilon}_{p} V^\phi(x)$       \\
			\hline
			$\forall \phi : \Prob \left( System^\phi \vDash Spec \right) = 1$
		\end{tabular}
		\caption{Proof rule of \textit{LexPSM}.}
	\end{figure}
\end{center}

Let us define
$ith^{\phi,\Pi} : \mathbb{N}_{<d} \times \mathbb{N}_{<2^h} \to \wp(S \times Q) \times ((S \times Q) \times \mathbb{N}_{<d} \to \mathbb{R}^d)$
as the function that maps an index to a pair of a partition of the state space and the associated \textit{LexPSM}
\begin{equation}
	ith^{\phi,\Pi}(p, i) = (P_{p,i}, \lambda_{p,i})
\end{equation}

$V^{\phi,\Pi} : (S \times Q) \times \mathbb{N}_{<d} \to \mathbb{R}^d$:
\begin{equation}
	V^{\phi,\Pi}(x, p) = \begin{cases}
		\lambda^\phi_{p,0}(x)     & \text{if } x \in P^\phi_{p,0}     \\
		\vdots                                                        \\
		\lambda^\phi_{p,2^h-1}(x) & \text{if } x \in P^\phi_{p,2^h-1}
	\end{cases}
\end{equation}

the $\Post_{(g, U)} : (S \times Q) \times \mathbb{N}_{<d} \to (\mathbb{R}^d)_{n \in \mathbb{N}_{<2^{|U|+h}}} $ operator for \textit{TreeLexPSM} is defined as:
\begin{equation}
	\begin{split}
		Posts_U = \prod_{(p,u) \in U} & \{ p \cdot \lambda^\phi_{p,i}(u(x)) \mid \\
		& \quad i \in \mathbb{N}_{<2^h} P^\phi_{p,i}, \lambda^\phi_{p,i} = ith^{\phi,\Pi}(p, i) P^\phi_{p,i} \cap u(x) \ne \emptyset \}
	\end{split}
\end{equation}
\begin{equation}
	\Post_{(g, U)} V^{\phi,\Pi}(x, p) = \left\{ \sum_{x' \in Posts_u} x' \mid Posts_u \in  Posts_U \right\}
\end{equation}

Now we can rewrite the proof rule for the synthesis of a \textit{TreeLexPSM}:
\begin{center}
	\begin{figure}[h!]
		\begin{tabular}{l}
			$\forall \phi : I(q_0, Init^\phi) \ge 0$                                                                                                                                                      \\
			$\forall \phi \forall x \in S \times Q \forall (g, U) \in \mathcal{G}^\phi_\times \forall u \in U :$                                                                                          \\
			\qquad $TI(x) \land I(x) \ge 0 \land g(x) \Implies I(u(x)) \ge 0$                                                                                                                             \\
			$\forall \phi \forall \Even p \in \mathbb{N}_{<d} \forall x \in S \times Q \forall (g, U) \in \mathcal{G}^\phi_\times \forall i \in \mathbb{N}_{<2^h}:$                                       \\
			\qquad $O_p(x) \land TI(x) \land I(x) \ge 0 \land g(x) \land P^\phi_{p,i}(x) \Implies \bigvee_{post \in \Post_{(g, U)} V^{\phi,\Pi}(x, p)} post \preceq^{\varepsilon}_{p} V^{\phi,\Pi}(x, p)$ \\
			$\forall \phi \forall \Odd p \in \mathbb{N}_{<d} \forall x \in S \times Q \forall (g, U) \in \mathcal{G}^\phi_\times \forall i \in \mathbb{N}_{<2^h} :$                                       \\
			\qquad $O_p(x) \land TI(x) \land I(x) \ge 0 \land g(x) \land P^\phi_{p,i}(x) \Implies \bigvee_{post \in \Post_{(g,U)} V^{\phi,\Pi}(x, p)} post \prec^{\varepsilon}_{p} V^{\phi,\Pi}(x,p)$     \\
			\hline
			$\forall \phi : \Prob \left( System^\phi \vDash Spec \right) = 1$
		\end{tabular}
		\caption{Proof rule of \textit{TreeLexPSM}.}
	\end{figure}
\end{center}

\section{Future Work}

For the time being a \textit{TreeLexPSM} has $d$ binary decision trees each one of height $h$;
one possible improvement would be to allow different heights for each tree.
This refinement would need further studies in order to be able to understand which of the binary
trees requires a greater expressivity (height) or if it's the overall invariant $I$ that has to
be refined in an CEGIS based guess-check algorithm implementation.

\end{document}
